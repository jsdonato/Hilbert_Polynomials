\documentclass[11pt]{amsart} 
\usepackage{amssymb} 
\usepackage{enumerate} 
\usepackage{hyperref} 
\usepackage{tikz} 
\usepackage{fancyhdr}
\usepackage{mathtools}
\usepackage[thinc]{esdiff}
\usepackage{amsmath}
\usepackage{graphicx}
\usepackage{mathrsfs}

\newcommand{\Z}{\mathbb{Z}}
\newcommand{\R}{\mathbb{R}}
\newcommand{\C}{\mathbb{C}}
\newcommand{\Q}{\mathbb{Q}}
\newcommand{\N}{\mathbb{N}}
\newcommand{\I}{\mathbb{I}}
\newcommand{\Ra}{\Rightarrow}
\newcommand{\La}{\Leftarrow}
\newcommand{\ra}{\rightarrow}
\newcommand{\Lra}{\Longrightarrow}
\newcommand{\labs}{\left\lvert}
\newcommand{\rabs}{\right\rvert}
\newcommand{\LAP}{\mathscr{L}}
\DeclareMathOperator{\GL}{GL} 


\theoremstyle{plain} 
\newtheorem{theorem}{Theorem}[section]
\newtheorem{proposition}[theorem]{Proposition} 
\newtheorem{lemma}[theorem]{Lemma}
\newtheorem{corollary}[theorem]{Corollary}
\newtheorem{conjecture}[theorem]{Conjecture}


\theoremstyle{definition} 
\newtheorem{definition}[theorem]{Definition} 
\newtheorem{remark}[theorem]{Remark} 


\title{Conditons on the existence of $\lambda$ for Constant and Linear Hilbert Polynomials }
\author{Faustas Udrenas}

\begin{document}

\maketitle
\begin{theorem}[Zhang, Donato, Udrenas; (2020)]
Let $N>0$ and $H_I(d)$ be a Hilbert Polynomial in $N+1$ variables. 
\begin{itemize}
\item For $H_I(d)=K$ for some $K\in \R$, then $\lambda$ exists if and only if  $K\in \N$. $\lambda$ takes the form  
\[\lambda = (1^{[K]})\]
\item For $H_I(d)=Md-r$ for some $M,r\in \R, M\neq 0$ , $\lambda$ exists if and only if $M\in \N$, $r\in \Z$ and
\[r\leq \frac{M^2-3M}{2}\]
$\lambda$ takes the form 
\[\lambda = (2^{[M]},1^{[\frac{M^2-3M}{2}-r]})\]
\end{itemize}
\end{theorem}

\newpage

\textit{Proof for constant Linear Polynomials $H_I(d)=K$:}\\
"$\implies$"\\
Assume that $\lambda$ exists.\\
First we show that  for all $\lambda_i\in \lambda,\lambda_i=1$. Consider $\lambda_1$. By Macaulays theorem, this is the largest lambda value in the lambda partition. Assume $\lambda_1=r>1$ $r in \N$. Thus by Macaulays theorem, $H_I(d)$ contains the term
\[\binom{d+r-1}{r-1}= \frac{(d+r-1)!}{(r-1)!(d)!}= \frac{(d+r-1)\cdot(d+r-2)\cdot\hdots\cdot(d+1)}{(r-1)!}\]
But notice that this would imply that $H_I(d)$ contains a $d^{r-1}$ term, which is a contradiciton since $r-1\geq1$ and $H_I(d)=K$ is a constant polynomial. Thus we conclude $r\leq1$ but by Macaulys theorem, we have that $r=1$. Thus we have that $\lambda_i =1 \;\; \forall \lambda_i\in \lambda$ but notice that
\[\binom{d+\lambda_i-i}{\lambda_i-1} = \binom{d+1-i}{0}  = 1\]
Thus if $\lambda$ exists and $\lambda_i=1$ for all $\lambda_i\in\lambda$ then 
\[\sum_{i=1}^{\big|\lambda\big|}\binom{d+\lambda_i-1}{\lambda_i-1} = \big|\lambda\big|\cdot 1 =\big|\lambda\big|\]
where $\big|\lambda\big|$ is the number of $\lambda_i \in \lambda$
Thus this implies
\[K=\big|\lambda\big|\]
and since $\big|\lambda\big|\in \N$ we have that $K \in \N$. From this we conclude
\[\lambda=(1^{[K]})\]


"$\Longleftarrow$"\\
Assume $K\in \N$.\\
Then we can write $H_I(d)$ as
\[H_I(d)=1+1+\hdots+1 =K\]
and note
\[\binom{d+1-i}{1-1} =\binom{d+1-i}{0} = 1, \forall i \]
so
\[H_I(d)=\binom{d+1-1}{1-1}+\binom{d+1-2}{1-1}+\hdots+\binom{d+1-K}{1-1} = K \;\; \implies \lambda=(1^{[K]})\]
\hfill $\square$
\newpage
\textit{Proof for Linear Hilbert Polynomials $H_I(d)=Md-r$:}\\
$"\implies"$\\
Assume $\lambda$ exists. Now first we show that  for all $\lambda_i\in \lambda, \lambda_i =1$ or $\lambda_i=2$. Consider $\lambda_1$. By Macaulays theorem, this is the largest lambda value in the lambda partition. Assume $\lambda_1=F>2$, $F\in \N$. Thus by Macaulays theorem, $H_I(d)$ contains the term
\[\binom{d+F-1}{F-1}= \frac{(d+F-1)!}{(F-1)!(d)!}= \frac{(d+F-1)\cdot(d+F-2)\cdot\hdots\cdot(d+1)}{(F-1)!}\]
But this would imply that $H_I(d)=Md-r$ contains a $d^{F-1}$ term and $F-1\geq2$ which clearly is a contradiction. Thus $F\leq 2$. Now notice that for $\lambda_i=2$ tells us that the following term is in the sume that makes up $H_I(d)$
\[\binom{d+2-i}{2-1}=\binom{d+2-i}{1}=d+2-i\]
Recall that the term correpsonding to when $\lambda_i$ has no linear term in d, thus, the number of $2$'s in the $\lambda$ partition determine the coefficient of the linear term of the hilbert polynomial $H_I(d)$. Thus, $M\in \N$. \\
However notice that for $\lambda=(2^{M})$ we have that
\[H_I(d)= \sum_{i=1}^{M}\binom{d+2-i}{1}=\sum_{i=1}^{M}d+2-i=\sum_{i=1}^{M}d +\sum_{i=1}^{M}2-\sum_{i=1}^{M}i =Md+2M-\frac{(M)(M+1)}{2}\]
\[\implies H_I(d)= Md-\frac{M^2-3M}{2}\]
Thus notice that for any $1$'s in the $\lambda$-partition will only add $1$'s to this polynomial. Thus if $r> \frac{M^2-3M}{2}$ then $\lambda$ cannot exist, thus
\[r\leq \frac{M^2-3M}{2} \]
Additionally notice that $\frac{M^2-3M}{2}$ is always an integer for any $M\in \N$. Thus adding ones to $\frac{M^2-3M}{2}$ still results in an integer value. Thus $r$ is an integer.\\
%Lastly, in order for $H_I(d)$ to equal $Md-r$ we need there to be $\frac{M^2-M3}{2}$ number of ones in $\lambda$ so that
%\[Md-\frac{M^2-3M}{2}+\sum_{i=M+1}^{\frac{M^2-3M}{2}-r+M}\binom{d_1-i}{0} = Md-\frac{M^2-3M}{2}+(\frac{M^2-3M}{2}-r) = Md-r\]

\newpage

$"\Longleftarrow"$
Let $H_I(d)=Md-r$, $M\in \N$, $r\in \Z$ and
\[r\leq \frac{M^2-3M}{2}\]
Then choose 
$\lambda=(2^{[M]},1^{[\frac{M^2-3M}{2}-r]})$
Thus we get that
\[\lambda \;\; \implies \;\; \sum_{i=1}^{\frac{M^2-3M}{2}-r+M}\binom{d+\lambda_i-i}{\lambda_i-1} = \sum_{i=1}^{M}(d+2-i) + \sum_{i=M+1}^{\frac{M^2-3M}{2}-r+M}1 \]
\[= Md+\frac{M^2-3M}{2}+\frac{M^2-3M}{2}-r = Md-r\]
so for this Hilbert polynomial with these conditions, the $\lambda$ exists.\\
\vspace{1mm}
\hfill $\square$


\begin{conjecture}
Let $N=3$ and $H_I(d)$ be a Hilbert Polynomial in $N+1$ variables. \\
$\lambda$ exists if and only if all conditions below are satisfied
\begin{itemize}
\item $H_I(d)=ax^2+x+c$ for some $a,b\in \mathbb{Q}, c\in \mathbb{Z}$
\item a is a multiple of $\frac{1}{2}$ 
\item $b - (2a[2-a]) \in \mathbb{N}$
\item $c - \frac{1}{3}\big(4a^3-12a^2+11a\big)\geq  \Big[(2a^2-4a-b)(2-2a)-\Big(\frac{(2a^2-4a-b)(2a^2-4a-b+1)}{2}\Big)\Big]$

\end{itemize} 
\end{conjecture}



\end{document}