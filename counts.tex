\documentclass[11pt]{amsart} % Define the document type. The 11pt refers to font size.


%%%This is the preamble

%%%%%%%%%%%%%%%%%%%%%%%%%%%%%%
% Include packages required to compile document:

\usepackage{amssymb} % can access some special symbols with this package
\usepackage{enumerate} % This package is used to provide various types of
%numbering for lists.
\usepackage{hyperref} % This package allows you to refer within the
%document with clickable hyperblinks.
\usepackage{tikz} %This package is for creating graphics. I am including it
%because of the provided example in the figure. You may not need to include
%this package.


% You may need to add additional packages depending on the commands you
% want to use.

%%%%%%%%%%%%%%%%%%%%
% User defined commands:

\newcommand{\Z}{\mathbb Z} %shortcut for writing the integers using Z with
%blackboard font.
\renewcommand{\P}{\mathbb P} %shortcut for writing probability using P with
%blackboard font. Since \P is already defined, we have to renew the command.

% You may add shortcuts for any commands you find yourself using frequently.  

\DeclareMathOperator{\GL}{GL} % This command lets you declare math
%operators.  Some built in operators are \sin for sine and \cos for cosine.
%Math operators appear nicely in the math environment.  

% End user defined commands
%%%%%%%%%%%%%%%%%%%%%%%%%%%%%%%

%%%%%%%%%%%%%%%%%%%%%%%%%%%%%%%
% These establish different environments for stating Theorems, Lemmas,
% Remarks, etc. You can add to these but you should not change the existing
% ones.


\theoremstyle{plain} %theorems after this line but before the definition
%style will all have the same 'plain' style
\newtheorem{theorem}{Theorem}[section]
\newtheorem{proposition}[theorem]{Proposition} %the optional arugment
%ensures that propositions and theorems have the same counter.
\newtheorem{lemma}[theorem]{Lemma}
\newtheorem{corollary}[theorem]{Corollary}

\theoremstyle{definition} %theorems after this line will all have the same
%'definition' style
\newtheorem{definition}[theorem]{Definition} 
\newtheorem{remark}[theorem]{Remark} 

%End environments
%%%%%%%%%%%%%%%%%%%%%%%%%%%%%%%%

%title page information
\title{LoG(M) Project Zero}
\author{Joseph Donato}

%%%%%%%%%%%%%%%%%%%%%%%%%%%%%%%%
% Now we're ready to start
%%%%%%%%%%%%%%%%%%%%%%%%%%%%%%%%

\begin{document}

%%% This is the document

\maketitle

\section*{2-D (n=1)}
In the case when $n=1$ we are attempting to count the number of saturated ideals for our Hilbert polynomial for when $\lambda=(1^{r})$ and our Hilbert polynomial becomes $H_{I}(d)=r$.\\
Our problem of counting the Ideals up to saturation then boils down to counting the number of ways of laying down $r$ rows along either the x-axis or y-axis.  To count this we simply consider the weak composition of $k$ things into $2$ buckets which gives us that the number of saturated ideals for when $n=1$ is 
$$\binom{r+2-1}{2-1}=\binom{r+1}{1}=r+1$$

\section*{3-D (n=2)}
Now when we consider the case when $n=1$, our lambda partition becomes $\lambda=(2^{p},1^{r})$ and our Hilbert polynomial is of the form $H_{I}(d)=kd+c$ where $k,c\in\mathbb{R}$ and must satisfy some conditions which we discussed in the past.  Further, to make our count simpler note that if we lay planes in any configuration along the $x-y,y-z,x-z$ planes it does not affect how we lay down rows in the future.  Keeping this in mind if we let $n_{2}(p)$ be the number of ways to lay down $p$ planes and $n_{1}(r)$ the number of ways to lay down $r$ rows then the total number of ways to lay down these objects becomes
$$n_{2}(p)n_{1}(r)$$
First, when considering $n_{2}(p)$ we just have to count the number of ways to place $p$ planes 3 places, these being the $x-y,y-z,x-z$ planes.  So then we have to consider a weak composition of $p$ objects into 3 buckets when finding a closed form for $n_{2}(p)$.  So now we have
$$n_{2}(p)=\binom{p+3-1}{3-1}=\binom{p+2}{2}$$
Now when finding a closed form for $n_{1}(r)$ lets first consider the number of ways to place $r$ rows along 3 different axis $x,y,z$.  This boils down to a weak composition of $r$ objects into 3 buckets which is
$$\binom{r+3-1}{3-1}=\binom{r+2}{2}$$
Now this only gives us the number of ways to place $r$ rows into 3 different slots.  However, within each slot there are additional ways to arrange these rows.  So if we have $c$ rows along a particular axis and we consider the projection of these rows onto the plane of the remaining two axis the number of ways t arrange those is $f_{2}(c)$ which is the number of young diagrams or integer partitions for the integer $c$.  So now we have that the total number of Ideals up to saturation for $\lambda=(2^{p},1^{r})$ is 
$$n_{2}(p)n_{1}(r)=\binom{p+2}{2}\sum_{i=1}^{\binom{r+2}{2}}\left[f_{2}(c_{x_{i}})f_{2}(c_{y_{i}})f_{2}(c_{z_{i}})\right]$$
Where $i$ is an index over all weak compositions of $r$ rows into 3 axes



\end{document}