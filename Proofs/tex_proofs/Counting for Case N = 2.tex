\documentclass[11pt]{amsart} % Define the document type. The 11pt refers to font size.
\setlength{\parskip}{1em}


%%%This is the preamble

%%%%%%%%%%%%%%%%%%%%%%%%%%%%%%
% Include packages required to compile document:

\usepackage{amssymb} % can access some special symbols with this package
\usepackage{enumerate} % This package is used to provide various types of
%numbering for lists.
\usepackage{hyperref} % This package allows you to refer within the
%document with clickable hyperblinks.
\usepackage{tikz} %This package is for creating graphics. I am including it
%because of the provided example in the figure. You may not need to include
%this package.


% You may need to add additional packages depending on the commands you
% want to use.
\usepackage{amssymb}
\usepackage{amsfonts}
\usepackage{bbm}

%%%%%%%%%%%%%%%%%%%%
% User defined commands:

\newcommand{\Z}{\mathbb Z} %shortcut for writing the integers using Z with
%blackboard font.
\renewcommand{\P}{\mathbb P} %shortcut for writing probability using P with
%blackboard font. Since \P is already defined, we have to renew the command.

% You may add shortcuts for any commands you find yourself using frequently.  

\DeclareMathOperator{\GL}{GL} % This command lets you declare math
%operators.  Some built in operators are \sin for sine and \cos for cosine.
%Math operators appear nicely in the math environment.  

% End user defined commands
%%%%%%%%%%%%%%%%%%%%%%%%%%%%%%%

%%%%%%%%%%%%%%%%%%%%%%%%%%%%%%%
% These establish different environments for stating Theorems, Lemmas,
% Remarks, etc. You can add to these but you should not change the existing
% ones.


\theoremstyle{plain} %theorems after this line but before the definition
%style will all have the same 'plain' style
\newtheorem{theorem}{Theorem}[section]
\newtheorem{proposition}[theorem]{Proposition} %the optional arugment
%ensures that propositions and theorems have the same counter.
\newtheorem{lemma}[theorem]{Lemma}
\newtheorem{corollary}[theorem]{Corollary}

\theoremstyle{definition} %theorems after this line will all have the same
%'definition' style
\newtheorem{definition}[theorem]{Definition} 
\newtheorem{remark}[theorem]{Remark} 

%End environments
%%%%%%%%%%%%%%%%%%%%%%%%%%%%%%%%

%title page information
\title{LoG(M) Project: \\ HILBERT SCHEMES AND MONOMIAL IDEALS \\ Counting (N = 2)}
\author{Zijian Zhang}
\date{March 23 2020}

%%%%%%%%%%%%%%%%%%%%%%%%%%%%%%%%
% Now we're ready to start
%%%%%%%%%%%%%%%%%%%%%%%%%%%%%%%%
\begin{document}
\maketitle
\thispagestyle{plain}

\section{Counting for n = 2 Case}
\begin{theorem}
\text{[1]} Given \(R = \mathbbm{C}[x_0, …, x_n]\) and polynomial \(p(d)\) in one variable, there exists ideals in R with Hilbert polynomial p if and only if \(p(d)\) can be written in the form \(\Sigma^{m}_{i = 1} \binom{d + \lambda_i - i}{\lambda_i - 1}\) for some \(n \geqslant \lambda_1 \geqslant ... \geqslant \lambda_m \geqslant 1\).
\end{theorem}

From the Theorem 1.1, we know that \(\lambda_{i} = 1\) or \(2\). When \(\lambda_{i} = 1\), the corresponding term in the Hilbert polynomial is \(\binom{d + 1 - i}{1 - 1} = \binom{d + 1 - i}{0} = 1\). And when \(\lambda_{i} = 2\), the corresponding term in the Hilbert polynomial is \(\binom{d + 2 - i}{2 - 1} = \binom{d + 2 - i}{1} = d + 2 - i\). Hence, all the Hilbert polynomials in case n = 2 are of the form \(p(d) = kd + c, where \) \(k, c \in \mathbbm{Z}.\)

Now, let's consider a given \(p(d) = kd + c\). Not like n = 1 case, where any Hilbert polynomial is of the form p(d) = c, p(d) now contains both a constant term and a linear term.

When it comes to the monomials in the ring \(\mathbbm{R}[x, y, z]\), they are all of the form \(x^ay^bz^c\). Suppose monomial \(x^{a_0}y^{b_0}z^{c_0}\) is within an ideal I of the ring \(\mathbbm{R}[x, y, z]\), then by the definition of the ideal, all of the monomials \(x^ay^bz^c\) s.t. \(a \geqslant a_0, b \geqslant b_0\) and \(c \geqslant c_0\) are also within I. Therefore, any ideal contains at least one (and maybe more than one) 3D sub-space(s). Hence, what is left outside of the ideal will no longer be a 3D space, but a set of 2-D planes and 1-D rows.

Now, again, consider the monomials in the ring \(\mathbbm{R}[x, y, z]\), they are all of the form \(x^ay^bz^c\). Note \(Degree(x^ay^bz^c) = a + b + c\). If we fixed a + b + c = d, then all of the monomials of degree d actually are within the same plane in the x-y-z space of monmials. Combining with the discovery of last paragraph, we get the main idea of counting for case n = 2 that planes contribute to the linear term and the constant term, while in the Hilbert polynomial and rows contribute to the constant term in the Hilbert polynomial.

To conclude, the counting of monomial ideals in n = 2 case, up to saturation, for any given Hilbert polynomial, is actually counting how to lay down a given number of planes and rows in 3-D space. And the \(\lambda\) partition provides exactly what we want, i.e. \(\lambda = (2^p, 1^r)\) corresponds to p 2-D planes and r 1-D rows.

Another key observation is that lying down any number of \((n - 1)-D\) spaces in \(n-D\) space will not affect the shape of the remaining n space. Thus, we can first lay down the planes and then lay down the rows.

For laying down the planes, we have three choices for each plane, i.e. parallel to x-y, y-z, or x-z plane. This is actually "Star and bars counting" such that there are p objects, and we need to place them into three different buckets. Then the total number of ways to lay down p planes is \(\binom{p + 3 - 1}{3 - 1} = \binom{p + 2}{2}\).

Then, for laying down the rows, we still have three sets of choices for each row, i.e. perpendicular to x-y, y-z, or x-z plane. Another key observation is that the rows perpendicular to one plane do not affect laying down rows perpendicular to another plane, this is because we only care about the pattern when d is large enough. Hence the number of ways to assign r objects to three groups is \(\binom{r + 3 - 1}{3 - 1} = \binom{r + 2}{2}\).

However, we can not simply treat this as a "Star and bars counting" problem, since the number of ways of laying down multiple rows within the same "group" (which means they are perpendicular to the same plane) is larger than one. Suppose we have in total k rows to lay down in the same "group", then we are actually constructing a Young diagram of size k. Hence, the number of ways doing so is \(f_2(k)\), which denotes the number of young diagram of size k.

Thus we need to consider \(\binom{r + 3 - 1}{3 - 1} = \binom{r + 2}{2}\) different cases. And for each case, we need to care about the number of possible young diagrams within each case:
$$\sum_{i=1, k_{x_i} + k_{y_i} + k_{z_i} = r}^{\binom{r+2}{2}}\left[f_{2}(k_{x_i})f_{2}(k_{y_i})f_{2}(k_{z_i})\right]$$

Note that \(k_{x_i}\) denotes the number of rows perpendicular to y-z plane, \(k_{y_i}\) denotes the number of rows perpendicular to x-z plane, and \(k_{z_i}\) denotes the number of rows perpendicular to x-y plane.

Now we could combine these two parts together by multiplying them since we are laying down planes and rows sequentially. Thus the final formula for the number of ways to lay down p planes and r rows is
$$\binom{p+2}{2}\sum_{i=1, k_{x_i} + k_{y_i} + k_{z_i} = r}^{\binom{r+2}{2}}\left[f_{2}(k_{x_i})f_{2}(k_{y_i})f_{2}(k_{z_i})\right]$$

\newpage
\thispagestyle{plain}
\begin{thebibliography}{9}
  \bibitem{Macaulay} 
    Francis Macaulay; {\it Some properties of enumeration in the theory of modular systems} Math. Soc. (2), 26, pp. 531-555 (1927)
\end{thebibliography}

\end{document}
